% Created 2019-09-23 Mon 20:59
% Intended LaTeX compiler: pdflatex
\documentclass[11pt]{article}
\usepackage[utf8]{inputenc}
\usepackage[T1]{fontenc}
\usepackage{graphicx}
\usepackage{grffile}
\usepackage{longtable}
\usepackage{wrapfig}
\usepackage{rotating}
\usepackage[normalem]{ulem}
\usepackage{amsmath}
\usepackage{textcomp}
\usepackage{amssymb}
\usepackage{capt-of}
\usepackage{hyperref}
\author{Dustin}
\date{\today}
\title{}
\hypersetup{
 pdfauthor={Dustin},
 pdftitle={},
 pdfkeywords={},
 pdfsubject={},
 pdfcreator={Emacs 26.3 (Org mode 9.2.4)}, 
 pdflang={English}}
\begin{document}

\tableofcontents



\section{Leatherman Homework \#2 (2019/09/21)}
\label{sec:org4eb83fc}
Do the following exercises from the Applied Linear Regression Models textbook. Each individual part (a, b, c, etc.) is worth 6.5 points.
    p. 91 -- \#2.10, \#2.12
        p. 90-93, \#2.4, \#2.13abc, \#2.23abc
        p. 92-93, \#2.16ab, \#2.26ab
\subsection{10 (p91)}
\label{sec:org5c725ed}
\emph{Explain whether a confidence or prediction interval is appropriate}
\subsubsection{a}
\label{sec:org735484d}
\emph{What will be the humidity level in this greenhouse tomorrow when we set the
temperature at \(31^\circ\) C?}

A prediction interval for a new observation is appropriate since the point of
interest is a single observation.
\subsubsection{b}
\label{sec:orgb406da2}
\emph{How much do families whose disposable income is \$23,500 spend, on the average,
for meals away from home?}

A confidence interval for a mean response is appropriate since the point of
interest is an average.
\subsubsection{c}
\label{sec:org21bcd76}
/How many kw-hours of electricty will be consumed next month by commercial and
industrial users in the Twin Cities service area, give that the index of usiness
activity for the area remains at its present level?/

A prediction interval for a new observation is appropriate since the point of
interest is a single observation.
\subsection{12 (p91)}
\label{sec:org59e8837}
/Can \(\sigma^2(pred)\) in 2.37 be brought increasingly close to 0 as n becomes
large? Is this also the case for \(\sigma^2(Y_h)\) in 22.9b? What is the
implication of this difference?/

As n becomes larger, Var(pred) approaches: \(Var(pred) = \sigma^2 + (X_h -
\bar{x})^2\)
This is not equivalent to 0.

As n increases, \(Var(\hat{Y_h}^2) = (X_h - \bar{X})^2\)

This means that as sample sizes grow, the variance of a prediction is equal to
the squared distance distance of a given predictor from its average. When \(X_h =
\bar{x}\), the variance of the prediction interval is equal to the variance of
\(Y_i\).


\subsection{Using the Grade Point Average Dataset}
\label{sec:orgc3d2a8a}
\subsubsection{4 (p90)}
\label{sec:org1d3cbcb}
\begin{enumerate}
\item a
\label{sec:org72f4a7a}
\emph{Obtain a 99\% C.I. for \(\beta_1\). Interpret. Does it include zero? Why might the
director of admissions be interested in whtether the C.I. includes zero?}

\begin{verbatim}
gpa <- readxl::read_excel("~/snap/firefox/common/Downloads/GradePointAverage.xlsx")
gpa.model1 <- lm(GPA ~ ACT, data = gpa)
confint(gpa.model1, level = 0.99)
# [0.0054, 0.0723]
\end{verbatim}

With 99\% confidence, each point scored on the ACT for a class of incoming
freshmen is associated with a mean increase in GPA between 0.0054 and 0.0723.

This confidence interval does not include zero which indicates that there is
convincing evidence that there is a relationship between a student's GPA and
their ACT score.
\item b
\label{sec:orgbb4a118}
/Test, using the test statistic t$\backslash$*, whether or not a linear association exists
between student's ACT score and GPA at the end o the freshman year Y. Use a lvel
of significane of 0.01. State the alternatives, decision rule, and conclusion./
\begin{verbatim}
summary(gpa.model1)
\end{verbatim}

Coefficients:
            Estimate Std. Error t value Pr(>|t|)
(Intercept)  2.11405    0.32089   6.588  1.3e-09 \textbf{*}
ACT          0.03883    0.01277   3.040  0.00292 **

\(H_0\): \(\beta_1 \eq 0\)

\(H_A\): \(\beta_1 \neq 0\)

There is convincing evidence that a linear association exists (\(\beta_1 \neq 0\)) between a
student's ACT score and their GPA at the end of their freshman year (p-value =
0.00292. two-sided t-test).
\item c
\label{sec:orgcd00731}
\emph{What is the p-value of your test in part b? How does it support the conclusion
reached in part b?}

The p-value is 0.00292. This is less than the significance level which indicates
that there is significant evidence that a relationship exists between GPA at the
end of freshman year and ACT score.
\end{enumerate}
\subsubsection{13 (p91)}
\label{sec:org70e5228}
\begin{enumerate}
\item a
\label{sec:org64f15fb}
\emph{Obtain a 95\% C.I. for students whose act test score is 28. Interpret}

\begin{verbatim}
predict(gpa.model1, data.frame(ACT=c(28)), interval = "confidence")
#        fit      lwr      upr
# 1 3.201209 3.061384 3.341033
\end{verbatim}

It is estimated that a student with an ACT score of 28 will have a GPA of 3.2 at
the end of their freshman year. With 95\% confidence, an ACT score of 28 is
associated with an average GPA between 3.0613 and 3.341 at the end of their freshman year.
\item b
\label{sec:org0f71e03}
\emph{Mary Jones obtained a score of 28 on the entrance test. PRedict her freshman
GPA using a 95\% P.I. Interpret.}

\begin{verbatim}
predict(gpa.model1, data.frame(ACT=c(28)), interval = "prediction")
#       fit      lwr      upr
# 1 3.201209 1.959355 4.443063
\end{verbatim}

Given that Mary Jones has an ACT score of 28, it is estimated that Mary Jones
will have a 3.2 GPA at the end of her freshman year. With 95\% confidence, her GPA
at the end of freshman year will be between 1.9593 and 4.443.
\item c
\label{sec:org5140d70}
As expected, the prediction interval is wider than the confidence interval. This
should always be the case.
\end{enumerate}
\subsubsection{23 (p93)}
\label{sec:org9171a64}
\begin{enumerate}
\item a
\label{sec:orgc766a8c}

\begin{verbatim}
anova(gpa.model1)
\end{verbatim}
\begin{center}
\begin{tabular}{lrrrr}
Source & SS & df & MS & F Statistic\\
\hline
Regression & 3.588 & 1 & 3.5878 & 9.2402\\
Error & 45.818 & 118 & 0.3883 & \\
\textbf{Total} & 49.406 & 119 &  & \\
\end{tabular}
\end{center}

\item b
\label{sec:orgc88a290}
The estimated MSR is 3.588. The MSE is 0.3883. When \(n = 3\) and \(\hat{Y_i} =
\bar{y}\), they will estimate the same quantity.
\item c
\label{sec:orgd49bdef}

\(H_0\): \(\beta_1 = 0\)

\(H_A\): \(\beta_1 \neq 0\)

The F statistic is 9.2402 which corresponds to a p-value of 0.002917. There is
convincing evidence that there is a relationship between ACT and GPA at the end
of freshman year.
\end{enumerate}

\subsection{Using the Plastic Hardness Dataset}
\label{sec:orgeabd5e6}
\subsubsection{16}
\label{sec:org6aec14d}
\begin{enumerate}
\item a
\label{sec:org3a42668}
\emph{Obtain a 98\% C.I. for the mean hardness of molded items with an elapse time of
30 hours. Interpret.}

\begin{verbatim}
plastic <- readxl::read_excel("~/Downloads/PlasticHardness.xlsx")
plastic.model1 <- lm(Hardness ~ Hours, data = plastic)
predict(plastic.model1, newdata = data.frame(Hours=c(30)), interval = "confidence", level = .98)
\end{verbatim}
       fit      lwr      upr
1 229.6312 227.4569 231.8056

It is estimated that the average hardness for a molded item that has aged for 30
hours is 229.6312. With 98\% confidence, a curation time of 30 hours is
associated with an average hardness of between 227.4569 and 231.8056.
\item b
\label{sec:orgffb9009}
\emph{Obtain a 98\% P.I. for the hardness of a newly molded test item with an elapsed
time of 30 hours.}

\begin{verbatim}
predict(plastic.model1, newdata = data.frame(Hours=c(30)), interval = "prediction", level = .98)
\end{verbatim}
       fit      lwr     upr
1 229.6312 220.8695 238.393

With 98\% confidence, a random mold that has curated for 30 hours will have
a hardness between 220.8695 and 238.393.
\end{enumerate}
\subsubsection{26}
\label{sec:org87324d8}
\begin{enumerate}
\item a
\label{sec:orgf35efd8}
\begin{verbatim}
anova(plastic.model1)
\end{verbatim}
\begin{center}
\begin{tabular}{lrrrr}
Source & SS & df & MS & F Statistic\\
\hline
Regression & 5297.5 & 1 & 5297.5 & 506.51\\
Error & 146.4 & 14 & 10.5 & \\
\textbf{Total} & 5443.9 & 15 &  & \\
\end{tabular}
\end{center}
\item b
\label{sec:org75d03aa}

\(H_0\): \(\beta_1 = 0\)

\(H_A\): \(\beta_1 \neq 0\)

The F statistic is 506.51 which corresponds to a p-value of 2.159e-12. There is
convincing evidence that there is a relationship between Hours cured and Hardness.
\end{enumerate}
\end{document}
